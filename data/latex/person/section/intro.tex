
%----------------------------------------------------------------------------------------
% Mini page taking up 66% of the actual page
\begin{minipage}[t]{\dimexpr0.5\linewidth-1em}

\begin{figure}[H]
\centering\includegraphics[width=0.4\linewidth]{figure/logo-aka.jpg}
\end{figure}

\textbf{Aikikai NSW Attendance System, 8th Nov 2020.}

Aikikai NSW is trialing a new web based class attendance system.
This new system will help us maintain more accurate attendance records,
reduce the administration burden to maintain these records,
and help to determine student-to-student contacts in the case of COVID-19 infection.
The system is being trialed by a few dojos in NSW. The trial will
be expanded once we are happy that the system is sufficiently stable and usable.

\smallskip
The new system is based on QR (Quick Response) codes, similar to those used
at other organizations and businesses to track attendance to their venues.

\medskip
We use a two step workflow:

\textbf{1. Student Registration}
\begin{center}
\includegraphics[scale=0.4]{figure/fig-student-registration.pdf}
\end{center}

\begin{enumerate}
\item The student (here, Taylor) uses their phone to scan their
      own personal \emph{student registration} QR code.

\item The student registration code directs the phone to the
      attendance server, which shows a page to register the phone
      as belonging to this student.

\item Once the student clicks a button on the registration page,
      the server provides a \emph{cookie} (a special code) that includes
      the Aikikai identification number of the student. The phone stores
      this cookie, and will give a copy back to the server if later requested.
\end{enumerate}

Each student should only need to perform the registration process once.
However, they may need to register again if they change phones,
or if they instruct the web browser on their phone to clear the cookies it is holding.

Your own personal registration QR code is provided on the other page.
If you lose these printed notes then your instructor can download a new PDF version
from the attendance site, including your registration code.
\end{minipage}
\hspace{2em}
%-----------------------------------------------------------
\begin{minipage}[t]{\dimexpr0.5\linewidth-1em}
\textbf{2. Class Registration}
\begin{center}
\includegraphics[scale=0.4]{figure/fig-class-registration.pdf}

\begin{enumerate}
\item When the student attends a class they use their phone to scan
      a different \emph{class registration} QR code.
\item The class registration code directs the phone to the
      attendance server.
\item The server requests a copy of the cookie provided previously,
      which it uses to determine which student is connecting to it.
\item Once the phone replies with the student cookie,
      the server will show the page to register this particular
      student for the class.
\end{enumerate}
\end{center}


\smallskip
%-----------------------------------------------------------
\textbf{3. Frequently Asked Questions}
\begin{enumerate}
\item \emph{Q: What if I don't have a phone, or don't bring it?}
      A: The class instructor can add your name to the attendance list
         directly, via a separate web page.

\item \emph{Q: Who has access to the attendance records?} \\
      A: Instructors and administrators that are part of Aikikai Australia.
         The physical computer that hosts the site is managed by a professional
         hosting company (Linode), located in Sydney, Australia. The hosting company
         also has access to the physical machine. The server does not run any other
         services besides the attendance system.

\item \emph{Q: Does clearing cookies from my phone affect my attendance records?}
      A: No. The cookie only contains your encoded Aikikai identification number.
         The cookie is used by the server to determine which student
         is connecting to it. The attendance records are stored on the server.

\item \emph{Q: What if I give my phone to someone else?} \\
      A: You can remove the cookie from your phone by either 1) using the ``clear cookies"
         functionality of your web browser, or 2) using the registration QR code to return
         to the registration page and clicking ``unregister". You can unregister
         and re-register a phone using your own code whenever you like.
\end{enumerate}


\medskip
%-----------------------------------------------------------
\textbf{4. Contact Details}

Please contact Ben Lippmeier \texttt{benl@ouroborus.net} for
any questions, concerns or feedback.
This is a new system, so we are interested in any feedback or ideas to make
it easier to use.
\end{minipage}
